%% LyX 2.3.6.1 created this file.  For more info, see http://www.lyx.org/.
%% Do not edit unless you really know what you are doing.
\documentclass[english]{article}
\usepackage[T1]{fontenc}
\usepackage[latin9]{inputenc}
\usepackage{geometry}
\geometry{verbose,tmargin=4cm,bmargin=4cm,lmargin=4cm,rmargin=4cm}
\usepackage{babel}
\usepackage{amstext}
\usepackage[unicode=true]
 {hyperref}

\makeatletter
%%%%%%%%%%%%%%%%%%%%%%%%%%%%%% Textclass specific LaTeX commands.
\newenvironment{lyxcode}
	{\par\begin{list}{}{
		\setlength{\rightmargin}{\leftmargin}
		\setlength{\listparindent}{0pt}% needed for AMS classes
		\raggedright
		\setlength{\itemsep}{0pt}
		\setlength{\parsep}{0pt}
		\normalfont\ttfamily}%
	 \item[]}
	{\end{list}}

\makeatother

\begin{document}
\title{EBA3500 Fall 2021 \\
Exercises 3: Inference, transformations, and correlation}
\author{Jonas Moss}
\maketitle

\section{Inference}

How do you define a \emph{p}-value, and how do you use them in regression?

\section{$R^{2}$ and the correlation}

\subsection*{a) The $R^{2}$ in least absolute deviations regression}

Make a function $\mathtt{rsq\_lad}$ that calculates the $R^{2}$
for least absolute deviations ($R_{lad}^{2}$ from now on). Make sure
to test it up against at least two examples. Is $R_{lad}^{2}$ symmetric,
i.e., is the $R_{lad}^{2}$ between $y$ and $x$ the same as the
$R_{lad}^{2}$ between $x$ and $y$?

\emph{Hint}: To avoid bugs due to naming of columns, use $data=pd.DataFrame({'y':y,'x':x})$
inside the regression function instead of $data=pd.DataFrame([y,x].T)$.
In very special cases, the progam calculating the least absolute deviations
will fail to give correct results. You have to return $\mathtt{np.max([0,rsq])}$to
make the $R^{2}$ correct. (This is equivalent to the slope being
$0$ and and the intercept equal to the median.) 

\subsection*{b) Anscombe's Quartet}

Anscombe's quartet are four dissimilar datasets with the same correlations
/ $R^{2}$s using least squares. Load the data and plot it using.
\begin{lyxcode}
import~seaborn~as~sns~

sns.set\_theme(style=\textquotedbl ticks\textquotedbl )

df~=~sns.load\_dataset(\textquotedbl anscombe\textquotedbl )

sns.lmplot(x=\textquotedbl x\textquotedbl ,~y=\textquotedbl y\textquotedbl ,~col=\textquotedbl dataset\textquotedbl ,~

hue=\textquotedbl dataset\textquotedbl ,~data=df,~col\_wrap=2,~

ci=None,~palette=\textquotedbl muted\textquotedbl ,~height=4,~scatter\_kws=\{\textquotedbl s\textquotedbl :~50,~\textquotedbl alpha\textquotedbl :~1\})~

plt.show()
\end{lyxcode}
Calculate the $R_{lad}^{2}$ values for these four datasets, both
$R_{lad}^{2}(x,y)$ and $R_{lad}^{2}(y,x)$. Compare the the least
squares case.

\emph{Hint:} You may want to use (and try to understand) 
\[
\mathtt{\text{df.groupby("dataset").apply(lambdaz:rsq\_lad(z["x"],z["y"]))}}
\]

Be sure to read the hint in (a). Try to reverse the axes of the plot
using x=\textquotedbl y\textquotedbl , y=\textquotedbl x\textquotedbl{}
inside sns.lmplot above.

\subsection*{c) Correlation using least absolute deviations}

Using the equivalent definition of the correlation (for least squares)
in the slides, propose a definition of the correlation for least absolute
deviations. Make a Python function that calculates this ``LAD-correlation''.
\begin{lyxcode}
\end{lyxcode}

\section{Non-linear least squares}

\subsection{Estimator--plotter function}

Make the following function:
\begin{lyxcode}
nls\_plotter(x,~y,~func):

''{}''{}''Estimate~the~parameters~of~func~(as~in~curve\_fit)~

using~non-linear~least~squares.~Make~a~scatterplot~of

x~and~y~and~add~the~curve~defined~by~func~using~the~estimated

parameters.~Return~the~non-linear~R~squared.''{}''{}''
\end{lyxcode}
\emph{Hint:} You may use $\mathtt{*params}$ to \href{https://docs.python.org/3/tutorial/controlflow.html\#unpacking-argument-lists}{unpack arguments}.

\subsection{Four parameterized functions}

Make the following functions in Python. Make sure they are compatible
with $\mathtt{curve\_fit}$. See \href{https://sites.ualberta.ca/~lkgray/uploads/7/3/6/2/7362679/slides_-_nonlinearlogisticregression.pdf}{this link for more.}.

\begin{eqnarray*}
\frac{a}{1+b^{(x-c)}}+d &  & \textrm{(Logistic curve)}\\
a+\frac{b}{x+c} &  & \textrm{(Hyperbolic curve)}\\
a+bx^{c} &  & \textrm{(Logarithmic curve)}\\
a+bc^{x} &  & \textrm{(Exponential curve)}
\end{eqnarray*}


\subsection{4-plotter}

Make a plotting function $\mathtt{grid\_plotter}$ that makes 4 separate
plots, one for each of the functions above, in a single window. The
plot contains a scatterplot of $y$ vs $x$ and adds the non-linear
least squares fitted curves for one of the functions above. The function
should return a dictionary with the $R^{2}$ for each of the curves.
Write a suitable docstring for the function.

\emph{Hint:} Use e.g. \href{https://towardsdatascience.com/master-the-art-of-subplots-in-python-45f7884f3d2e}{this}
to make a suitable plot.

\subsection{Application}

We will look at the dataset obtained from
\begin{lyxcode}
from~rdatasets~import~data~

dataset~=~data(\textquotedbl DNase\textquotedbl )~

dataset~=~dataset{[}dataset{[}'Run'{]}~==~1{]}

x~=~np.log(dataset{[}\textquotedbl conc\textquotedbl{]})~

y~=~dataset{[}\textquotedbl density\textquotedbl{]}
\end{lyxcode}
Use the $\mathtt{grid\_plotter}$ function on this data. Which functional
form fits the data best?
\end{document}
